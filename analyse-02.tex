\documentclass[1Opt]{report}
\author{Louis Merlin}
\usepackage[utf8]{inputenc}
\usepackage{amsfonts}
\usepackage{amsmath}

\renewcommand{\chaptername}{Chapitre}

\begin{document}

\title{Analyse II}
\maketitle

\tableofcontents


\chapter{Équations différentielles ordinaires}

\section{Définitions et exemples}


\paragraph{Exemple 1}
\[y'=0 \Rightarrow y(x)=C \quad \mbox{où} \quad C\in{\mathbb R}\]
$y(x)=2$ est une solution, et $y(x)=C,\,\forall C\in{\mathbb R}$ est une solution
plus générale.

\paragraph{Définition}
Une équation différentielle ordinaire est une expression
\[ E(x,y(x),y'(x),\ldots,y^{(n)}(x))=0\]
où $E:{\mathbb R}^{n+2}\rightarrow{\mathbb R}$ une fonction donnée,
$n\in{\mathbb R}^\ast$\\
On cherche un intervalle ouvert $I\subset{\mathbb R}$ et une fonction
$y:I\rightarrow{\mathbb R}$ de classe $C^n$ telle que l'équation soit satisfaite
pour tout $x\in I$.

\paragraph{Applications}
\begin{description}
  \item EDO $\rightarrow$ croissance de la population, désintegration radioactive.
  \item EDP $\rightarrow$ prévisions météo, marché financier.
\end{description}

\paragraph{Exemple 2}
\begin{eqnarray*}
   y''=0 & \Rightarrow & y'(x)=C_1\quad \mbox{ pour} \quad C_1\in{\mathbb R}, x\in{\mathbb R} \\
   y'=C & \Rightarrow & y(x)=C_1x+C_2\quad \mbox{ pour} \quad C_1,C_2\in{\mathbb R},x\in{\mathbb R}
\end{eqnarray*}

\paragraph{Exemple 3}
\[y+y'=0\Rightarrow y=y'\]
Rappel : $(a^x)'=a^x\log{a}, \, \forall a\in{\mathbb R}$
\[\log{a}=-1\Rightarrow a=\frac{1}{e}\Rightarrow \left(\left(\frac{1}{e}\right)^x\right)'=-\left(\frac{1}{e}\right)^x\]
$(e^{-x})'=-e^{-x}$ est une solution pour tout $x\in{\mathbb R}$.\\
Plus généralement, $(Ce^{-x})'=-Ce^{-x}$ pour $C\in{\mathbb R},\,x\in{\mathbb R}$.

\paragraph{Exemple}
$y'=-y$\quad "Équation à variables séparées" \\
Si on écrit $y'=\frac{dy}{dx}\Rightarrow\frac{dy}{dx}=-y$
\begin{eqnarray*}
  \begin{tabular}{c c}
    $\frac{dy}{y}=-dx$ & $\int{\frac{dy}{y}}=-\int{dx}$ \\
    variables séparées & les primitives
  \end{tabular}
\end{eqnarray*}
\begin{eqnarray*}
  & \Rightarrow & \log{|y|}=-x+C_1 \\
  & \Rightarrow & |y|=e^{-x+C_1}=e^{C_1}\times e^{-x}\\
  & \Rightarrow & |y|=C_2e^{-x},\, C_2>0\\
  & \Rightarrow & y(x)=\pm C_2e^{-x},\, C_2>0
\end{eqnarray*}
Mais $y(x)=0$ est aussi une solution.\\
Finalement on a : $y(x)=Ce^{-x},\,\forall C\in{\mathbb R},\,\forall x\in{\mathbb R}$.

\paragraph{Terminologie}
\begin{equation*}
  E\left(x,y(x),y'(x),\ldots,y^{(n)}\right)=0 \tag{$\ast$}
\end{equation*}

\paragraph{Définition}
L'ordre de l'équation $(\ast)$ est $n$ si $E$ est une fonction non-constante
de $y^{(n)}$

\paragraph{Définition}
Si $(\ast)$ est une expression polynomiale de $y^{(n)}$, alors le degré de
l'équation est le degré du polynôme en $y^{(n)}$. Si le degré est 1, alors
l'équation est dite linéaire.

\paragraph{Définition}
Si l'expression $(\ast)$ ne dépend pas de de x, l'équation différentielle est
dite autonome.

\end{document}
