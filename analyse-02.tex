\documentclass[1Opt]{report}
\author{Louis Merlin}
\usepackage[utf8]{inputenc}
\usepackage{amsfonts}
\usepackage{amsmath}
\usepackage{amssymb}

\renewcommand{\chaptername}{Chapitre}

\begin{document}

\title{Analyse II}
\maketitle

\tableofcontents


\chapter{Équations différentielles ordinaires}

\section{Définitions et exemples}


\paragraph{Exemple 1}
\[y'=0 \Rightarrow y(x)=C \quad \mbox{où} \quad C\in{\mathbb R}\]
$y(x)=2$ est une solution, et $y(x)=C,\,\forall C\in{\mathbb R}$ est une solution
plus générale.

\paragraph{Définition}
Une \textbf{équation différentielle ordinaire} est une expression
\[ E(x,y(x),y'(x),\ldots,y^{(n)}(x))=0\]
où $E:{\mathbb R}^{n+2}\rightarrow{\mathbb R}$ une fonction donnée,
$n\in{\mathbb R}^\ast$\\
On cherche un intervalle ouvert $I\subset{\mathbb R}$ et une fonction
$y:I\rightarrow{\mathbb R}$ de classe $C^n$ telle que l'équation soit satisfaite
pour tout $x\in I$.

\paragraph{Applications}
\begin{description}
  \item EDO $\rightarrow$ croissance de la population, désintegration radioactive.
  \item EDP $\rightarrow$ prévisions météo, marché financier.
\end{description}

\paragraph{Exemple 2}
\begin{eqnarray*}
   y''=0 & \Rightarrow & y'(x)=C_1\quad \mbox{ pour} \quad C_1\in{\mathbb R}, x\in{\mathbb R} \\
   y'=C & \Rightarrow & y(x)=C_1x+C_2\quad \mbox{ pour} \quad C_1,C_2\in{\mathbb R},x\in{\mathbb R}
\end{eqnarray*}

\paragraph{Exemple 3}
\[y+y'=0\Rightarrow y=y'\]
Rappel : $(a^x)'=a^x\log{a}, \, \forall a\in{\mathbb R}$
\[\log{a}=-1\Rightarrow a=\frac{1}{e}\Rightarrow \left(\left(\frac{1}{e}\right)^x\right)'=-\left(\frac{1}{e}\right)^x\]
$(e^{-x})'=-e^{-x}$ est une solution pour tout $x\in{\mathbb R}$.\\
Plus généralement, $(Ce^{-x})'=-Ce^{-x}$ pour $C\in{\mathbb R},\,x\in{\mathbb R}$.

\paragraph{Exemple}
$y'=-y$\quad "Équation à variables séparées" \\
Si on écrit $y'=\frac{dy}{dx}\Rightarrow\frac{dy}{dx}=-y$
\begin{eqnarray*}
  \begin{tabular}{c c}
    $\frac{dy}{y}=-dx$ & $\int{\frac{dy}{y}}=-\int{dx}$ \\
    variables séparées & les primitives
  \end{tabular}
\end{eqnarray*}
\begin{eqnarray*}
  & \Rightarrow & \log{|y|}=-x+C_1 \\
  & \Rightarrow & |y|=e^{-x+C_1}=e^{C_1}\times e^{-x}\\
  & \Rightarrow & |y|=C_2e^{-x},\, C_2>0\\
  & \Rightarrow & y(x)=\pm C_2e^{-x},\, C_2>0
\end{eqnarray*}
Mais $y(x)=0$ est aussi une solution.\\
Finalement on a : $y(x)=Ce^{-x},\,\forall C\in{\mathbb R},\,\forall x\in{\mathbb R}$.

\paragraph{Terminologie}
\begin{equation*}
  E\left(x,y(x),y'(x),\ldots,y^{(n)}\right)=0 \tag{$\ast$}
\end{equation*}

\paragraph{Définition}
L'\textbf{ordre} de l'équation $(\ast)$ est $n$ si $E$ est une fonction non-constante
de $y^{(n)}$

\paragraph{Définition}
Si $(\ast)$ est une expression polynomiale de $y^{(n)}$, alors le \textbf{degré} de
l'équation est le degré du polynôme en $y^{(n)}$. Si le degré est 1, alors
l'équation est dite linéaire.

\paragraph{Définition}
Si l'expression $(\ast)$ ne dépend pas de de x, l'équation différentielle est
dite \textbf{autonome}.

\paragraph{Remarque}
Si l'équation différentielle est autonome, et
$y(x)={\mathbb R}\rightarrow{\mathbb R}$ est une solution, alors $y(x+C)$ l'est
aussi pour tout $C\in{\mathbb R},\,x\in{\mathbb R}$ (par exemple dans l'Exemple
3 : $y(x)=Ce^{-x}\Rightarrow y(x)=Ce^{-(x+C')}$ est aussi une solution).

\paragraph{Type des équations différentielles}
\begin{eqnarray*}
  \begin{tabular}{l c c c}
    Equation & Ordre & Degré & Autonome \\
    $y'=0$ & 1 & 1 & oui \\
    $y''=0$ & 2 & 1 & oui \\
    $y+y'=5x+1$ & 1 & 1 & non \\
    $sin(y')=0$ & 1 & $\varnothing$ & oui \\
    $e^x(y')^2+y=0$ & 1 & 2 & non
  \end{tabular}
\end{eqnarray*}

\paragraph{Définition}
La \textbf{solution générale} d'une équation différentielle est l'ensemble de
toutes les solutions de l'équation.

\paragraph{Exemple}
$y'=0\Rightarrow y(x)=2$ est une solution sur ${\mathbb R}$, mais ce n'est pas
la solution générale. La solution générale est $y(x)=C$ pour tout
$C\in{\mathbb R}$.

\paragraph{Définition}
\textbf{Problème de Cauchy}\\
Résoudre l'équation $E\left(x,y(x),y'(x),\ldots,y^{(n)}\right)=0$ et trouver
l'intervalle ouvert $I\subset{\mathbb R}$ et une fonction
$y(x):I\rightarrow{\mathbb R}$ de classe $C^n(I)$ telle que
$y(x_0)=b_0,\,y(x_1)=b_1,\, etc$. Le nombre de conditions initiales dépend du
type de l'équation.

\paragraph{Exemple}
$y''=0\Rightarrow$ la solution générale est
$y(x)=C_1x+C_2,\,\forall C_1,C_2\in{\mathbb R}$ sur ${\mathbb R}$. \\
Si on a : $y(0)=1$ et $y(2)=4$ comme conditions initiales, alors $y(0)=C_2=1$ et
$y(2)=C_1\times 2+C_2=2C_1+1=4\Rightarrow C_1=\frac{3}{2}$.\\
La solution particulière satisfaisant les conditions initiales est $y(x)=
\frac{3}{2}x+1$.

\section{Equations différentielles à variables séparées (EDVS) (du premier ordre)}

\paragraph{Définition}
$f(y)\times y'(x)=g(x)$ où
\begin{tabular}{l}
  $f:I\rightarrow{\mathbb R}$ est une fonction continue sur $I$ \\
  $g:I\rightarrow{\mathbb R}$ est une fonction continue sur $J$ \\
\end{tabular}
est une \textbf{équation différentielle à variables séparées} (EDVS).

\paragraph{Explication}
\[ f(y)\frac{dy}{dx}=g(x)\Leftrightarrow\int{f(y)dy}=\int{g(x)dx}\]
Une fonction $y:J'\subset J\rightarrow I$ qui satisfait l'équation est une
solution de classe $C'$.

\paragraph{Théorème}
\textbf{Existence et unicité d'une solution de EDVS} \\
Soit
\begin{tabular}{l}
  $f:I\rightarrow{\mathbb R}$ une fonction continue telle que $f(y)\neq0$ sur $I$. \\
  $g:I\rightarrow{\mathbb R}$ une fonction continue.
\end{tabular} \\
Alors pour tout couple $(x_0\in J,\,b_0\in I)$ l'équation
\begin{equation*}
  f(y)y'(x)=y(x) \tag{$\ast \ast$}
\end{equation*}
admet une solution $y:J'\rightarrow I$ vérifiant les conditions initiales
$y(x_0)=b_0$. \\
Si $y_1:J_1\rightarrow I$ et $y_2:J_2\rightarrow I$ sont deux solutions telles
que $y_1(x_0)=y_2(x_0)=b_0$, alors $y_1(x)=y_2(x)$ pour $x\in J_1\cap J_2$.

\paragraph{Démonstration}
Soit :
\begin{eqnarray*}
  F(y)=\int_{b_0}^y f(s)ds & \Rightarrow & F(y) \mbox{ est monotone} \\
  & \Rightarrow & F(y) \mbox{ est inversible et } F(b_0)=0 \\
  & \Rightarrow & f(y)\neq0 \mbox{ sur } I
\end{eqnarray*}
Soit :
\[ G(x)=\int^x_{x_0}g(s)ds\rightarrow G(x_0)=0 \quad x_0,x\in J \]
Soit $y(x)=F^{-1}(G(x))$ sur un voisinage de $x_0\in J$ :
\begin{eqnarray*}
  \Rightarrow & F(y(x))=G(x) \\
  \Rightarrow & F'(y(x))\times y'(x)=G'(x) \\
  \Rightarrow & f(y(x))\times y'(x)=g'(x) \\
  \Rightarrow & y(x)=F^{-1}(G(x)) \mbox{ est une solution} \\
  \Rightarrow & y(x_0)=F^{-1}(G(x_0))=F^{-1}(0)=b_0 \\
  \Rightarrow & y(x) \mbox{ satisfait les conditions initiales} \\
\end{eqnarray*}
\underline{Unicité :} Soient $y_1(x)$ et $y_2(x)$.
\begin{eqnarray*}
  y_1(x_0)=y_2(x_0)=b_0 & \Rightarrow & F(y_1(x))=F(y_2(x)) \\
  & \Rightarrow & y_1(x)=y_2(x)
\end{eqnarray*}

\end{document}
