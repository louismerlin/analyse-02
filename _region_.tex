\message{ !name(analyse-02.tex)}\documentclass[1Opt]{report}
\author{Louis Merlin}
\usepackage[utf8]{inputenc}
\usepackage{amsfonts}
\usepackage{amsmath}
\usepackage{amssymb}

\renewcommand{\chaptername}{Chapitre}

\begin{document}

\message{ !name(analyse-02.tex) !offset(-3) }


\title{Analyse II}
\maketitle

\tableofcontents


\chapter{Équations différentielles ordinaires}

\section{Définitions et exemples}


\paragraph{Exemple 1}
\[y'=0 \Rightarrow y(x)=C \quad \mbox{où} \quad C\in{\mathbb R}\]
$y(x)=2$ est une solution, et $y(x)=C,\,\forall C\in{\mathbb R}$ est une solution
plus générale.

\paragraph{Définition}
Une \textbf{équation différentielle ordinaire} est une expression
\[ E(x,y(x),y'(x),\ldots,y^{(n)}(x))=0\]
où $E:{\mathbb R}^{n+2}\rightarrow{\mathbb R}$ une fonction donnée,
$n\in{\mathbb R}^\ast$\\
On cherche un intervalle ouvert $I\subset{\mathbb R}$ et une fonction
$y:I\rightarrow{\mathbb R}$ de classe $C^n$ telle que l'équation soit satisfaite
pour tout $x\in I$.

\paragraph{Applications}
\begin{description}
  \item EDO $\rightarrow$ croissance de la population, désintegration radioactive.
  \item EDP $\rightarrow$ prévisions météo, marché financier.
\end{description}

\paragraph{Exemple 2}
\begin{eqnarray*}
   y''=0 & \Rightarrow & y'(x)=C_1\quad \mbox{ pour} \quad C_1\in{\mathbb R}, x\in{\mathbb R} \\
   y'=C & \Rightarrow & y(x)=C_1x+C_2\quad \mbox{ pour} \quad C_1,C_2\in{\mathbb R},x\in{\mathbb R}
\end{eqnarray*}

\paragraph{Exemple 3}
\[y+y'=0\Rightarrow y=y'\]
Rappel : $(a^x)'=a^x\log{a}, \, \forall a\in{\mathbb R}$
\[\log{a}=-1\Rightarrow a=\frac{1}{e}\Rightarrow \left(\left(\frac{1}{e}\right)^x\right)'=-\left(\frac{1}{e}\right)^x\]
$(e^{-x})'=-e^{-x}$ est une solution pour tout $x\in{\mathbb R}$.\\
Plus généralement, $(Ce^{-x})'=-Ce^{-x}$ pour $C\in{\mathbb R},\,x\in{\mathbb R}$.

\paragraph{Exemple}
$y'=-y$\quad "Équation à variables séparées" \\
Si on écrit $y'=\frac{dy}{dx}\Rightarrow\frac{dy}{dx}=-y$
\begin{eqnarray*}
  \begin{tabular}{c c}
    $\frac{dy}{y}=-dx$ & $\int{\frac{dy}{y}}=-\int{dx}$ \\
    variables séparées & les primitives
  \end{tabular}
\end{eqnarray*}
\begin{eqnarray*}
  & \Rightarrow & \log{|y|}=-x+C_1 \\
  & \Rightarrow & |y|=e^{-x+C_1}=e^{C_1}\times e^{-x}\\
  & \Rightarrow & |y|=C_2e^{-x},\, C_2>0\\
  & \Rightarrow & y(x)=\pm C_2e^{-x},\, C_2>0
\end{eqnarray*}
Mais $y(x)=0$ est aussi une solution.\\
Finalement on a : $y(x)=Ce^{-x},\,\forall C\in{\mathbb R},\,\forall x\in{\mathbb R}$.

\paragraph{Terminologie}
\begin{equation*}
  E\left(x,y(x),y'(x),\ldots,y^{(n)}\right)=0 \tag{$\ast$}
\end{equation*}

\paragraph{Définition}
L'\textbf{ordre} de l'équation $(\ast)$ est $n$ si $E$ est une fonction non-constante
de $y^{(n)}$

\paragraph{Définition}
Si $(\ast)$ est une expression polynomiale de $y^{(n)}$, alors le \textbf{degré} de
l'équation est le degré du polynôme en $y^{(n)}$. Si le degré est 1, alors
l'équation est dite linéaire.

\paragraph{Définition}
Si l'expression $(\ast)$ ne dépend pas de de x, l'équation différentielle est
dite \textbf{autonome}.

\paragraph{Remarque}
Si l'équation différentielle est autonome, et
$y(x)={\mathbb R}\rightarrow{\mathbb R}$ est une solution, alors $y(x+C)$ l'est
aussi pour tout $C\in{\mathbb R},\,x\in{\mathbb R}$ (par exemple dans l'Exemple
3 : $y(x)=Ce^{-x}\Rightarrow y(x)=Ce^{-(x+C')}$ est aussi une solution).

\paragraph{Type des équations différentielles}
\begin{eqnarray*}
  \begin{tabular}{l c c c}
    Equation & Ordre & Degré & Autonome \\
    $y'=0$ & 1 & 1 & oui \\
    $y''=0$ & 2 & 1 & oui \\
    $y+y'=5x+1$ & 1 & 1 & non \\
    $sin(y')=0$ & 1 & $\varnothing$ & oui \\
    $e^x(y')^2+y=0$ & 1 & 2 & non
  \end{tabular}
\end{eqnarray*}

\paragraph{Définition}
La \textbf{solution générale} d'une équation différentielle est l'ensemble de
toutes les solutions de l'équation.

\paragraph{Exemple}
$y'=0\Rightarrow y(x)=2$ est une solution sur ${\mathbb R}$, mais ce n'est pas
la solution générale. La solution générale est $y(x)=C$ pour tout
$C\in{\mathbb R}$.

\paragraph{Définition}
\textbf{Problème de Cauchy}\\
Résoudre l'équation $E\left(x,y(x),y'(x),\ldots,y^{(n)}\right)=0$ et trouver
l'intervalle ouvert $I\subset{\mathbb R}$ et une fonction
$y(x):I\rightarrow{\mathbb R}$ de classe $C^n(I)$ telle que
$y(x_0)=b_0,\,y(x_1)=b_1,\, etc$. Le nombre de conditions initiales dépend du
type de l'équation.

\paragraph{Exemple}
$y''=0\Rightarrow$ la solution générale est
$y(x)=C_1x+C_2,\,\forall C_1,C_2\in{\mathbb R}$ sur ${\mathbb R}$. \\
Si on a : $y(0)=1$ et $y(2)=4$ comme conditions initiales, alors $y(0)=C_2=1$ et
$y(2)=C_1\times 2+C_2=2C_1+1=4\Rightarrow C_1=\frac{3}{2}$.\\
La solution particulière satisfaisant les conditions initiales est $y(x)=
\frac{3}{2}x+1$.

\section{Equations différentielles à variables séparées (EDVS) (du premier ordre)}

\paragraph{Définition}
$f(y)\times y'(x)=g(x)$ où
\begin{tabular}{l}
  $f:I\rightarrow{\mathbb R}$ est une fonction continue sur $I$ \\
  $g:I\rightarrow{\mathbb R}$ est une fonction continue sur $J$ \\
\end{tabular}
est une \textbf{équation différentielle à variables séparées} (EDVS).

\paragraph{Explication}
\[ f(y)\frac{dy}{dx}=g(x)\Leftrightarrow\int{f(y)dy}=\int{g(x)dx}\]
Une fonction $y:J'\subset J\rightarrow I$ qui satisfait l'équation est une
solution de classe $C'$.

\paragraph{Théorème}
\textbf{Existence et unicité d'une solution de EDVS} \\
Soit
\begin{tabular}{l}
  $f:I\rightarrow{\mathbb R}$ une fonction continue telle que $f(y)\neq0$ sur $I$. \\
  $g:I\rightarrow{\mathbb R}$ une fonction continue.
\end{tabular} \\
Alors pour tout couple $(x_0\in J,\,b_0\in I)$ l'équation
\begin{equation*}
  f(y)y'(x)=y(x) \tag{$\ast \ast$}
\end{equation*}
admet une solution $y:J'\rightarrow I$ vérifiant les conditions initiales
$y(x_0)=b_0$. \\
Si $y_1:J_1\rightarrow I$ et $y_2:J_2\rightarrow I$ sont deux solutions telles
que $y_1(x_0)=y_2(x_0)=b_0$, alors $y_1(x)=y_2(x)$ pour $x\in J_1\cap J_2$.

\paragraph{Démonstration}
Soit :
\begin{eqnarray*}
  F(y)=\int_{b_0}^y f(s)ds & \Rightarrow & F(y) \mbox{ est monotone} \\
  & \Rightarrow & F(y) \mbox{ est inversible et } F(b_0)=0 \\
  & \Rightarrow & f(y)\neq0 \mbox{ sur } I
\end{eqnarray*}
Soit :
\[ G(x)=\int^x_{x_0}g(s)ds\rightarrow G(x_0)=0 \quad x_0,x\in J \]
Soit $y(x)=F^{-1}(G(x))$ sur un voisinage de $x_0\in J$ :
\begin{eqnarray*}
  \Rightarrow & F(y(x))=G(x) \\
  \Rightarrow & F'(y(x))\times y'(x)=G'(x) \\
  \Rightarrow & f(y(x))\times y'(x)=g'(x) \\
  \Rightarrow & y(x)=F^{-1}(G(x)) \mbox{ est une solution} \\
  \Rightarrow & y(x_0)=F^{-1}(G(x_0))=F^{-1}(0)=b_0 \\
  \Rightarrow & y(x) \mbox{ satisfait les conditions initiales} \\
\end{eqnarray*}
\underline{Unicité :} Soient $y_1(x)$ et $y_2(x)$.
\begin{eqnarray*}
  y_1(x_0)=y_2(x_0)=b_0 & \Rightarrow & F(y_1(x))=F(y_2(x)) \\
  & \Rightarrow & y_1(x)=y_2(x)
\end{eqnarray*}

\paragraph{Définition}
La \textbf{solution maximale} de l'EDVS avec la condition initiale $y(x_0)=b_0$,
$x_0\in J$ et $b_0\in I$, est une fonction $y(x)$ de classe $C'$ qui satisfait
l'équation, la condition initiale, et qui est définie sur le plus grand
intervalle possibe.\\
Le théorème sur les EDVS dit qu'il existe une solution maximale de l'EDVS pour
la condition initiale $y(x_0)=b_0$ pour tout couple $x_0\in J,\,b_0\in I$.\\
Toute autre solution $y(x)$ qui satisfait $y(x_0)=b_0$ est une restriction de
la solution maximale.

\paragraph{Exemple 1}
\[ \frac{y'(x)}{y^2(x)}=1 \quad \mbox{EDVS: }f(y)=\frac{1}{y^2} \]
Pour donner la solution générale :
\begin{eqnarray*}
  \int{\frac{dy}{y^2}}=\int{dx} & \Rightarrow & -\frac{1}{y}=x+C,\,\forall C\in{\mathbb R} \\
  & \Rightarrow & y=-\frac{1}{x+C},\,\forall C\in{\mathbb R},\,x\neq -C
\end{eqnarray*}
Supposons qu'on cherche une solution telle que $y(0)=b_0$
\begin{eqnarray*}
  y(0)=\frac{1}{C}=b_0 & \Rightarrow & C=-\frac{1}{b_0} \\
  & \Rightarrow & y(x)=-\frac{1}{x-\frac{1}{b_0}}=\frac{b_0}{1-xb_0} \\
\end{eqnarray*}
\begin{eqnarray*}
  1-xb_0>0 & \Leftrightarrow & \frac{1}{b_0}>x \\
  1-xb_0<0 & \Leftrightarrow & \frac{1}{b_0}<x
\end{eqnarray*}



\section{Equations différentielles linéaires du premier ordre (EDL1)}


\paragraph{Définition}
Soit $I\subset{\mathbb R}$ un intervalle ouvert. Une équation de la forme
$y'(x)+p(x)\cdot y(x)=f(x)$, où $p,f:I\rightarrow{\mathbb R}$ sont continues est
une équation différentielle linéaire du premier ordre (EDL1). La solution est
une fonction $y:I\rightarrow{\mathbb R}$ de classe $C'$ qui satisfait
l'équation.

\paragraph{Proposition}
Considérons l'équation $y'(x)+p(x)y=0$ \\
Alors la fonction $y(x)=Ce^{-P(x)}, y:I\rightarrow{\mathbb R}$ est la solution
générale de cette équation pour tout $C\in{\mathbb R}$. \\
Ici $P(x)$ est une primitive de $p(x)$ sur $I$.

\paragraph{Démonstration}
\begin{eqnarray*}
  y'(x)=-p(x)y & \Rightarrow & \frac{y'(x)}{y(x)}=-p(x) \quad (EDVS) \\
  & \Rightarrow & \int{\frac{dy}{y}}=\int{-p(x)dx} \\
  & \Rightarrow & log|y|=P(x)+C_1 \quad C_1\in{\mathbb R} \\
  & \Rightarrow & |y|=e^{-P(x)+C_1}=e^{C_1}\cdot e^{-P(x)} \\
  & \Rightarrow & y(x)=Ce^{-P(x)},\,C=Ie^{C_1} \\
\end{eqnarray*}
Mais y(x)=0 est une solution, ainsi $y(x)=Ce^{-P(x),\,x\in I,\, C\in{\mathbb R}}$
est la solution générale de l'équation $y'(x)+p(x)y=0$.

\paragraph{Remarque}

\subparagraph{Vérification}
\[ y'(x)+p(x)y=Ce^{-P(x)}(-p(x))+p(x)Ce^{-P(x)}=0 \]
On a : $y'(x)+p(x)y(x)=0$ \\
Ainsi la solution générale est $y(x)=Ce^{-P(x)},\,C\in{\mathbb R},\,x\in I$ où
$P(x)$ est une primitive de $p(x)$.

\subparagraph{Principe de superposition des solutions}
Soit $I\subset{\mathbb R}$ un intervalle ouvert, $f_1,f_2:I\rightarrow{\mathbb R}$
continues. \\
Supposons que $v_1(x):I\rightarrow{\mathbb R}$ et $v_2(x):I\rightarrow{\mathbb R}$
sont des solutions particulières des équations
\begin{eqnarray*}
  & y'+p(x)y=f_1(x)\rightarrow v_1(x) &\\
  \mbox{et} & y'+p(x)y=f_2(x)\rightarrow v_2(x) & \mbox{respectivement.}
\end{eqnarray*}
Alors la fonction $v(x)=v_1(x)+v_2(x)$ est une solution particulière de
l'équation $y'+p(x)y=f_1(x)+f_2(x)$.

\subparagraph{Vérification}
\begin{eqnarray*}
  v'(x)+p(x)v(x) & = & (v_1'(x)+v_2'(x))+p(x)(v_1(x)+v_2(x))\\
  & = & v_1'(x)+p(x)v_1(x)+v_2'(x)+p(x)v_2(x)\\
  & = & f_1(x)+f_2(x)
\end{eqnarray*}

\subparagraph{Méthode de la variation des constantes}
On cherche une solution particulière de $y'(x)+p(x)y(x)=f(x),\quad
p,f:I\rightarrow$ continues.\\
(Ansatz) On cherche une solution de la forme
\[v(x)=C(x)\cdot e^{-P(x)} \quad \mbox{où P(x) est une primitive de p(x).}\]
Alors on obtient :
\begin{eqnarray*}
  & & v'(x)+p(x)v(x)=f(x)\\
  & \Rightarrow & C'(x)e^{-P(x)}-p(x)\cdot C(x)e^{-P(x)}+p(x)\cdot C(x)e^{-P(x)}=f(x)\\
  & \Rightarrow & C'(x)e^{-P(x)}=f(x)\\
  & \Rightarrow & C'(x)=f(x)e^{P(x)}\\
  & \Rightarrow & C(x)=\int{f(x)e^{P(x)}dx}
\end{eqnarray*}
Une situation particulière de l'équation $y'+p(x)y=f(x)$ est
$v(x)=(\int{f(x)e^{P(x)}dx})e^{-P(x)}$, où $P(x)$ est une primitive de $p(x)$,
et $v(x):I\rightarrow{\mathbb R}$.

\paragraph{Exemple 1}
\[y'+y=5x+1\quad \mbox{EDL1: }p(x)=1,\;f(x)=5x+1,\;p,f:\mathbb{R}\rightarrow\mathbb{R}\]
\[P(x)=\int{1dx}=x \quad\mbox{(une primitive)}\]
La solution générale de $y'+y=0$ est :
\[y(x)=Ce^{-P(x)}=Ce^{-x}\quad C\in\mathbb{R},\, x\in\mathbb{R}\]
Ainsi :
\begin{eqnarray*}
  C(x) & = & \int{f(x)e^{P(x)}dx}=\int{(5x+1)e^xdx}\\
  & = & \int{5xe^xdx}+\int{e^xdx}=\int{5xd(e^x)}+\int{e^xdx}\\
  & = & 5xe^x-5\int{e^xdx}+\int{e^xdx}=5xe^x-4\int{e^xdx}\\
  & = & 5xe^x-4e^x+C
\end{eqnarray*}
Une solution de $y'+y+5x+1$ est :
\[v(x)=(5xe^x-4e^x+C)e^{-x}=5x-4+Ce^{-x}\]

\subparagraph{Vérification}
\[v'(x)+v(x)=5-Ce^{-x}+5x-4+Ce^{-x}=5x+1\]
En effet c'est la solution générale.


\paragraph{Proposition}
Soient $f,p:I\rightarrow\mathbb{R}$ des fonctiosns continues. Supposons que
$v_0:I\rightarrow\mathbb{R}$ est une solution particulière de
$y'(x)+p(x)y(x)=f(x)$.\\
Alors la solution générale de cette équation est $v(x)=v_0(x)+Ce^{-P(x)}$ pour
tout $C\in\mathbb{R}$ où $P(x)$ est une primitive de $p(x)$ particulière.

\subparagraph{Démonstration}
Si $v_1(x)$ est une solution de $y'(x)+p(x)y=f(x)$, on a que $v_0(x)$ est une
solution aussi de la même équation.\\
Donc par le principe de superposition, la fonction $v_1(x)-v_0(x)$ est une
solution de $y'(x)+p(x)y(x)=0$ (sans second membre).\\
Ainsi $v_1(x)-v_0(x)$ est de la forme $Ce^{-P(x)}\quad C\in\mathbb{R}$, $P(x)$
une primitive de $p(x)$.\\
Ainsi la solution est $v_1(x)=v_0(x)+Ce^{-P(x)}$.

\paragraph{Exemple 2}
\[y'+y=5x+1\]
Solution associée sans second membre : $y(x)=Ce^{-x},\,C\in\mathbb{R}$.\\
Solution particulière : $y(x)=5x-4$.\\
Solution générale : $y(x)=5x-4+Ce^{-x},\,x\in\mathbb{R}$.


\subsection{Equation de Bernoulli}

\paragraph{Définition}
Une équation différentielle de la forme
\[y'+p(x)y=q(x)y^\alpha\quad p,q:I\rightarrow\mathbb{R}\mbox{ fonctions continues}\quad\alpha\in\mathbb{R},\alpha\neq0,\alpha\neq1\]
est dite \textbf{équation de Bernoulli}.\\
On peut transformer en EDL1 par le changemeent de variables :
\[z(x)=(y(x))^{1-\alpha}\]
Alors $z'(x)=(1-\alpha)(y(x))^{-\alpha}y'(x)=(1-\alpha)\frac{y'}{y^\alpha}$\\
$\frac{y'}{y^\alpha}+p(x)y^{1-\alpha}=q(x)$  (vérifier si $y(x)=0$ est une
solution de l'équation originale).
Ainsi $\frac{1}{1-\alpha}z'+y(x)z=q(x)$  c'est une EDL1.

\paragraph{Exemple 1}
é\[y'=\frac{4}{x}+x\sqrt{y}\Leftrightarrow y'-\frac{4}{x}y=x\sqrt{y}\]
C'est une équation de Bernoulli avec $\alpha=\frac{1}{2},y(x)\geq0,x\neq0$.
Changement de variable : $z(x)=(y(x))^{1-\frac{1}{2}}=\sqrt{y(x)}$.
$z'(x)=\frac{1}{2\sqrt{y}}y'$ en supposant $y(x)\neq0$.
\begin{eqnarray*}
 \frac{y'}{\sqrt{y}}-\frac{4}{x}\sqrt{y}=x & \Rightarrow & 2z'-\frac{4}{x}z=x
\quad\mbox{EDL1}\\
 & \Rightarrow & z'-\frac{2}{x}z=\frac{x}{2}
\end{eqnarray*}

\subparagraph{(1)}
On cherche la solution générale de l'équation associée ssm : $z'-\frac{2}{x}z=0$.
\begin{eqnarray*}
  p(x)=-\frac{2}{x} & \Rightarrow & P(x)=\int{-\frac{2}{x}dx}=-2\log{|x|}=
-\log{x^2},\,x\neq0\\
  & \Rightarrow & y_{ssm}=Ce^{-P(x)}=Ce^{\log{x^2}}=Cx^2,\,x\neq0
\end{eqnarray*}
\textbf{Vérification :} $(Cx^2)'-\frac{2}{x}(Cx^2)=2Cx-2Cx=0\quad\checkmark$

\subparagraph{(2)}
On cherche une solution particulière de l'équation complète.
\begin{eqnarray*}
C(x)&=&\int{f(x)e^{P(x)}dx}=\int{\frac{x}{2}e^{-\log{x^2}}dx}\\
&=&\int{\frac{x}{2}\frac{1}{x^2}dx}=\int{\frac{1}{2}\frac{dx}{x}}=\frac{1}{2}
\log{|x|},\,x\neq0
\end{eqnarray*}
(on supprime la constante).
\[z_{part}(x)=\frac{1}{2}\log{|x|}e^{\log{x^2}}=x^2\frac{1}{2}\log{|x|}
,\,x\neq0\]
\textbf{Vérification :} Soit $x<0\Rightarrow z_{part}(x)=x^2\frac{1}{2}\log{-x}$
\begin{eqnarray*}
  z'-\frac{2}{x}z&=&(\frac{x^2}{2}\log{-x})'-\frac{2}{x}(\frac{x^2}{2}\log{(-x)})\\
  &=&x\log{(-x)}+\frac{x^2}{2}\frac{1}{-x}(-1)-x\log{(-x)}=\frac{x}{2}
\end{eqnarray*}
On vérifie d'une manière similaire la solution pour $x>0$.
\textbf{Solution générale :}
\[z(x) = \left\{
  \begin{array}{lr}
    Cx^2+\frac{x^2}{2}\log{x}\, , & x\in]0,\infty[],\,C\in\mathbb{R}\\
    Cx^2+\frac{x^2}{2}\log{-x}\, , & x\in]-\infty,0[,\,C\in\mathbb{R}\\
  \end{array}
\right.
\]

\end{document}

\message{ !name(analyse-02.tex) !offset(-396) }
